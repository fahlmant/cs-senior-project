\documentclass{article}

\newcommand{\namesigdate}[2][5cm]{%
  \begin{tabular}{@{}p{#1}@{}}
    #2 \\[2\normalbaselineskip] \hrule \\[0pt]
    {\small \textit{Signature}} \\[2\normalbaselineskip] \hrule \\[0pt]
    {\small \textit{Date}}
  \end{tabular}
}

\usepackage[T1]{fontenc}
\usepackage[letterpaper, portrait, margin=0.75in]{geometry}
\usepackage[singlespacing]{setspace}
\usepackage{url}
\usepackage{tocloft}
\usepackage{listings}
\usepackage{color}
\usepackage{pgfgantt}
\setlength{\parindent}{0pt}

\begin{document}
\title {Progress Report for Noctilucent VR}
\author {Taylor Fahlman, Joshua Bowen, Adam Puckette}

\maketitle

\section{Project Recap}

Noctilucent VR will allow users to view and manipulate point-cloud data in virtual reality. 
It will be a free and open source software solution that draws upon several existing open source frameworks in order to display point cloud data via most virtual reality headsets. 
In addition, Noctilucent VR will accept user in order to measure and manipulate the point-cloud data from within the virtual reality environment. 
Our goals for Noctilucent VR are as follows:

\begin{enumerate}
\item To run on the Windows operating system
\item To display point-cloud data in a wide range of virtual reality headsets
\item To allow the measurement and manipulation of said data in real time
\item To display said data at a framerate and detail level suitable for virtual reality
\end{enumerate}

\section{Project Status}

\section{Problems}

\section{Weekly Activity Summary}
\subsection{Week 1}

\subsection{Week 2}

\subsection{Week 3}

\subsection{Week 4}

\subsection{Week 5}

\subsection{Week 6}

\subsection{Week 7}

\subsection{Week 8}

\subsection{Week 9}

\subsection{Week 10}

\section{Retrospective}

\begin{table}[ht]
\caption{Noctilucent VR Retrospective}
\centering
\begin{tabular}{l l l}
\hline\hline
Positives & Deltas & Actions \\ [0.5ex]
\hline
Content & Goes & Here \\
Just & Like & This \\ [1ex]
\hline
\end{tabular}
\end{table}

\end{document}
