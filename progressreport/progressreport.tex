\documentclass{article}

\newcommand{\namesigdate}[2][5cm]{%
  \begin{tabular}{@{}p{#1}@{}}
    #2 \\[2\normalbaselineskip] \hrule \\[0pt]
    {\small \textit{Signature}} \\[2\normalbaselineskip] \hrule \\[0pt]
    {\small \textit{Date}}
  \end{tabular}
}

\usepackage[T1]{fontenc}
\usepackage[letterpaper, portrait, margin=0.75in]{geometry}
\usepackage[singlespacing]{setspace}
\usepackage{url}
\usepackage{tocloft}
\usepackage{listings}
\usepackage{color}
\usepackage{pgfgantt}
\setlength{\parindent}{0pt}

\begin{document}
\title {Progress Report for Noctilucent VR}
\author {Taylor Fahlman, Joshua Bowen, Adam Puckette}

\maketitle

\section{Project Recap}

Noctilucent VR will allow users to view and manipulate point-cloud data in virtual reality. 
It will be a free and open source software solution that draws upon several existing open source frameworks in order to display point cloud data via most virtual reality headsets. 
In addition, Noctilucent VR will accept user in order to measure and manipulate the point-cloud data from within the virtual reality environment. 
Our goals for Noctilucent VR are as follows:

\begin{enumerate}
\item To run on the Windows operating system
\item To display point-cloud data in a wide range of virtual reality headsets
\item To allow the measurement and manipulation of said data in real time
\item To display said data at a framerate and detail level suitable for virtual reality
\end{enumerate}

\section{Project Status}

\subsubsection{Fall Term}

As of the end of Fall Term, Noctilucent VR has barely begun. 
We have checked and re-checked our design, and picked out the software frameworks that work best for the intended use of our project. 
With any luck, we will begin Winter Term prepared to start on the project without further delay. 
Our current choice of point-cloud viewing software is CloudCompare, though this may change in future depending on our needs. 
Potree Viewer is an attractive alternative, but we will not know which suits our needs more until we do a closer examination next term. 
OSVR is currently our virtual reality solution of choice, and barring any unforeseen developments in technology, it will remain ideal to our purposes. 
Our current status is: highly prepared.

\subsubsection{Winter Term: Week 6}

We are currently at the end of week 6. 
As of right now we are stuck on the OpenGL demo portion of our project.
For more details view the problems section.
We are curretnly working on creating an OpenGL VR Point Cloud viewer.
To accomplish this task we are planning to take an existing OpenGL VR program and insert our cloud viewer code.
Additionally to this, we are also working on modifying the Oculus VR support that exists inside of Cloud Compare to be more robust and support OSVR.
On top of that we are also looking into WebGL as a potential alternative to Cloud Compare.

\section{Problems}

\subsubsection{Fall Term}

As work has not yet begun we have not yet run into any problems, however there are some potential problems on the horizon that will need to be addressed.
The main problem that we forsee in the future is whether or not CloudCompare will be capable of rendering large point clouds in VR.
Several other people who have attempted a similair project with cloud compare have found that CloudCompare showed stress and rendering issues with VR.
Because of this issue we will need to put time into figuring out if CloudCompare is even capable of handling this project.
In the event that CloudCompare is incapable of handling this issue we will need to do one or two things.
The first option is we implement Octree datastructures in CloudCompare.
This solution could work as other, not Open Source, cloud viewing softwares utilize Octrees and have superior performance.
The problem with this solution is that it could be timely.
An alternate solution is to create an auxillery program that operates seperately from CloudCompare.
Ideally neither of these approaches will be neccessary but it is something that we need to think about.

\subsubsection{Winter Term: Week 6}

As was mentioned in the Winter Term Week 6 section, we are run into a couple issues with the OpenGL demo we wanted to make.
Unfortunately VR is a very poorly documented field, this includes peopels custom made demos.
The demo's people have made require a number of dependencies to actually work, but the process of installing them all and just trying to get everything to work is proving quite troublesome.
At this point we really just need to move on from this demo because it's sucking up a ton of time and we could be doing something better elsewhere.

\section{Weekly Activity Summary}
\subsection{Fall Week 3}

On our first week, we met our client (Matt O'Banion) and were introduced to the project and its goals. 
We viewed the current setup, and discussed what the end result would look like. We set up weekly meetings at a time that worked for all of us (1400 on Tuesdays), and have stuck to that time since. 
We were introduced to an issue that may come up to plague us later, namely that CloudCompare may not be capable of rendering large-scale point clouds at a framerate that works for VR. 
CloudCompare's creator offered to work with us to remedy this issue, so we shall see.

\subsection{Fall Week 4}

Nothing whatsoever of note occurred this week, aside from writing an abstract for our project and brainstorming good names. 
We had a slight scheduling hiccup near the week's end, but we addressed it and do not anticipate more problems in future. 
Mr. O'Banion was out-of-town this week, so we decided not to meet.

\subsection{Fall Week 5}

We began writing the Requirements Document this week, and met with Mr. O'Banion to discuss a schedule of deliverables for the term and align our schedules with his. 
We learned of several alternatives to the proposed software frameworks that we could use if necessary.

\subsection{Fall Week 6}

Once we had finished interpreting the IEEE-1998 standard, we were able to finish up the Requirements Document and planned to start in on the Technology Review on week 7. 
The testing setup in the Geomantics lab began acting up when we tried to get the positional tracking camera to work with it.

\subsection{Fall Week 7}

We divided up roles for the Technology Review, and researched our alternatives throughout most of the week. 
Mr. O'Banion was quite helpful, and pointed out a number of lesser-known and in-development programs that we probably would not have found otherwise. 
Once the Technology Review was submitted, we began deciphering the IEEE-1016-2009 standard. 
It took nearly twice as long to interpret as the IEEE-1998 standard, as it was quite a bit longer and had a number of extraneous sections that needed to be weeded out. 

\subsection{Fall Week 8}

We laid an outline of the Design Document this week, and set up the required section headers. 
I managed to isolate the IR tracking issue and confirm that the hardware was not at fault.

\subsection{Fall Week 9}

This week Mr. O'Banion left for New Zealand to run Lidar scans on the results of the recent earthquake, and that combined with the long Thanksgiving weekend meant that very little got done.

\subsection{Fall Week 10}

We finished up the Design Document at long last. The actual writing took very little time at all, once we knew what we were doing. 
Mr. O'Banion was still in New Zealand, and was unavailable to sign the Design Document. 
We began the Progress report on Friday, eager to be done for the term.

\subsection{Winter Week 1}

We met early in the week to try and get all the VR headsets working. 
We were partially successful in this effort, but are still unable to run anything but the official OSVR Palace Demo.
We also began working on getting a workable OpenGL demo.
Josh tried to create something himself but it didn't ultimately end up being as simple as previously thought.
From here we looked into already exisitng demos.

\subsection{Winter Week 2}

We started looking into some engines and things would be neccessary to get a working demo.
We began to have serious doubts as to the feasibility of CloudCompare due to the “second pass” it uses when rendering point clouds.
We also began some working figuring out how we wanted the UI to be structured.

\subsection{Winter Week 3}

This week we found some OpenGL demos that exist inside of OSVR already.
The next steps became looking into how to compile and get these demos working.
The goal was that once we got these demos working we would inject our code and it would work.
On top of this we set up the dev environment for Cloud Compare.

\subsection{Winter Week 4}

This week we spent just trying to get something to compile.
Our goal was to create a working cube in VR.
We began to consider WebGL as one of our possible options due to the WebVR project. 
It would be far easier for the user to setup than CloudCompare.

\subsection{Winter Week 5}

We continued working on trying to get a working demo.
There was a family emergency this week and one of our members needed to go home, so we were a bit stalled this week.

\subsection{Winter Week 6}

This week we continued trying to get something to compile.
We also worked on our Progress Report, Revisions, and Presentation.

\section{Retrospective}

\begin{table}[ht]
\caption{Noctilucent VR Retrospective}
\centering
\begin{tabular}{l l l}
\hline\hline
Positives & Deltas & Actions \\ [0.5ex]
\hline
Made clean problem statement & Get practice with OSVR & Finish OSVR Demo \\
GitHub set up & & \\
Decided on requirements & & \\
Decided on technologies used & & \\
Created design for future &  & \\
Met with client weekly & & \\
Experienced the current setup of 3D TV & & \\
Experienced OSVR headset with small demo & & \\
Divided the project into parts & & \\ [1ex]
\hline
\end{tabular}
\end{table}

\end{document}
