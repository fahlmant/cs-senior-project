\documentclass[titlepage]{article}

\usepackage[T1]{fontenc}
\usepackage[letterpaper, portrait, margin=0.75in]{geometry}
\usepackage[singlespacing]{setspace}
\usepackage{url}
\usepackage{tocloft}
\usepackage{listings}
\usepackage{color}
\setlength{\parindent}{0pt}

\begin{document}
\begin{titlepage}
\centering
{\Huge Requirements for Noctilucent VR\par}
{\Large Taylor Fahlman, Joshua Bowen, Adam Puckette}
\abstract
There are 4 main deliverables for this project. Creating an OpenGL test for the OSVR framework;
writing a proposal to either use Cloud Compare or other software; a simple documentation/summary of
the final product; and the final product, where an OSVR headset can interact with the software.
\end{titlepage}

\section{Introduction}
\subsection{Purpose}

The purpose of this addon is to allow any user of the open source software Cloud Compare to be able to view point cloud data in Virtual Reality.
The user, however, must be in possession of some Virtual Reality headware. 
In this regard this project is designed for anyone who utilizes Cloud Compare for it's open source cloud data viewing capabilities,
and is looking for a more immersive analysis of this data.

\subsection{Scope}

Noctilucent VR will function as an extension or addon of the already existent open source software Cloud Compare.
When utilized, this software will take the point cloud data as rendered by Cloud Compare and display it in a fashion suitable for virtual reality.
The software will render and update the display of the data in real time.
Auxillery devices will also be usable to navigate the data.
Benefits of this software include the viewing of point cloud data as it was intended to be viewed, in 3D.
Our goal is to make this software as easy to use and keep a consistent framerate.
While viewing the data the software should not stutter,
should draw complete scenes,
and should render millions of points.
As point cloud data often contain millions of points it is important that this software is able to render as many points as possible.

\subsection{Definitions}

Free and Open Source Software: Also known as FOSS, is any software that can be classified as both free software and open source software. This means that anyone is freely licensed to modify, copy, or study the code as they see fit. users are encouraged to improve the software and share their work with others.

Virtual Reality: Also referred to as VR, is an existing technology which allows the user to put on a headware and have an immersive experience in which turning your head turns the camera and moving your head moves the camera.
Additionally, visuals are rendered seperatly for both eyes allowing for different perspectives which when put together create a 3D environment.

Cloud Compare: Also referred to as CC, is a FOSS program freely available to users who need to view point cloud data.

Open Source Virtual Reality: Also referred to as OSVR, is an existing Open Source VR headset and framework. OSVR is designed so that it can be developed on and modified by the the public.

Oculus Developers Kit 2: Also called the DK2, is another existing VR headset, however it is not open source and has since been replaced with a standard production model.

\subsection{References}



\subsection{Overview}



\section{Overall Desciption}
\subsection{Product Perspective}



\subsection{Product Functions}

\subsection{User Characteristics}

\subsection{Constraints}

\subsection{Assumptions and Dependencies}

It is our assumption that CC will be capable of rendering a million or more points of cloud data in realtime.
In the event that CC is not capable of such we will need to considering creating a seperate software instead of an extension.

\section{Specific Requirements}


\end{document}

