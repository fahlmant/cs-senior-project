\documentclass[draftclsnofoot,onecolumn]{IEEEtran}

\newcommand{\namesigdate}[2][5cm]{%
  \begin{tabular}{@{}p{#1}@{}}
    #2 \\[2\normalbaselineskip] \hrule \\[0pt]
    {\small \textit{Signature}} \\[2\normalbaselineskip] \hrule \\[0pt]
    {\small \textit{Date}}
  \end{tabular}
}

\usepackage[T1]{fontenc}
\usepackage[letterpaper, portrait, margin=0.75in]{geometry}
\usepackage[singlespacing]{setspace}
\usepackage{url}
\usepackage{listings}
\usepackage{color}
\setlength{\parindent}{0pt}

\begin{document}
\title {Progress Report for Noctilucent VR}
\author {Taylor Fahlman, Joshua Bowen, Adam Puckette}

\maketitle

\abstract
This is the final

\section{Introduction}

Noctilucent VR is a WebVR-based plugin for the point-cloud viewing application, Potree Viewer. 
This plugin adds a virtual reality (VR) mode to the application which can be activated at the press of a button. 
The VR mode renders the point cloud in a 3D stereoscopic display suitable for VR headsets such as the Google Cardboard. 
This project was proposed by Matt O'Banion of the Geomatics laboratory at OSU in order to explore a cheaper, more portable and accessible alternative to their existing 3D TV setup. 
Mr. O'Banion also wished to experiment with the immersive aspects of a virtual-reality headset in the pursuit of his work with point cloud data sets. 
The potential of this project is in its accessible and inexpensive nature. 
With ready access to point-cloud data through the internet, anyone with a reasonably powerful computer and compatible VR headset could view point-cloud data in a more immersive format. 

The Noctilucent VR team consists of Joshua Bowen, Adam Puckette, and Taylor Fahlman. 

\section{Requirements Document}

\section{Requirements Document Changes}

\section{Design Document}

\section{Tech Review}

\section{Blog Posts}

\section{Poster}

\section{Project Documentation

\section{New Technology}

\section{Lessons Learned}

\section{Appendix #1: Code Listings}

\section{Appendix #2: Extras}

\end{document}
