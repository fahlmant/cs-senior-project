\documentclass{article}

\newcommand{\namesigdate}[2][5cm]{%
  \begin{tabular}{@{}p{#1}@{}}
    #2 \\[2\normalbaselineskip] \hrule \\[0pt]
    {\small \textit{Signature}} \\[2\normalbaselineskip] \hrule \\[0pt]
    {\small \textit{Date}}
  \end{tabular}
}

\usepackage[T1]{fontenc}
\usepackage[letterpaper, portrait, margin=0.75in]{geometry}
\usepackage[singlespacing]{setspace}
\usepackage{url}
\usepackage{tocloft}
\usepackage{listings}
\usepackage{color}
\usepackage{pgfgantt}
\setlength{\parindent}{0pt}

\begin{document}
\title {Requirements for Noctilucent VR}
\author {Taylor Fahlman, Joshua Bowen, Adam Puckette}

\maketitle

\abstract

\newpage
\thispagestyle{empty}
\mbox{}

\section{Overview}
\subsection{Scope}

This document describes the software design and information regarding the Virtual Reality Lidar Point Cloud Viewing Software known as Noctilucent VR.
This software is intended to be used by professionals seeking to improve their Lidar Cloud viewing experience.
Uses can include anything from casual cloud data viewing to professional informatics.

Noctilucent and it's design can be used on any computer capable of handling the graphics rendering.
This projects only applicability restriction is in the capabilities of the computer running it.

\subsection{Purpose}

Noctilucent VR exists to expand the current capabilities of humans to view Lidar Cloud data.
Specifically the goal of this project is to take 3D information previously displayed in 2D and provide an easy to use, easily accessible, Open Source alternative to viewing this data which allows for it's viewing in 3D.
This project exists as viewing 3D data in 3D is an important aspect of being able to garner easily all the information from that data as possible.

\subsection{Intended Audience}

This project is primarily intended to be used by scientists, engineers, and those who perform informatics, however, the project is not limited to those audiences.
The Free and Open Source nature of this project allows it's use by anyone with access to cloud compare, a virtual reality headset, a wiimote, and a computer capable of running this software.

\subsection{Conformance}

Noctilucent VR conforms to this document if it satisfies the requirements as they are outlined in Sections 4 and 5 of this Document. Requirements are denoted by the verb shall.

\section{Definitions}

\section{Conceptual Model}

\subsection{Introduction}
\subsection{SDD Idenfication}
\subsection{Design Stakeholders}
\subsection{Design Views}
\subsection{Design Viewpoints}
\subsection{Design Elements}
\subsection{Design Overlays}
\subsection{Design Rationale}
\subsection{Design Languages}

\section{Design Description}

\section{Design Viewpoints}

\subsection{Introduction}
\subsection{Context Viewpoints}
\subsection{Composition Viewpoint}
\subsection{Logical Viewpoint}
\subsection{Dependency Viewpoint}
\subsection{Information Viewpoint}
\subsection{Patterns use Viewpoint}
\subsection{Interface Viewpoint}
\subsection{Structure Viewpoint}
\subsection{Interaction Viewpoint}
\subsection{State Dynamics Viewpoint}
\subsection{Algorithm Viewpoint}
\subsection{Resource Viewpoint}

Annexs

\end{document}

