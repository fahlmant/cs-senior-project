\documentclass{article}

\newcommand{\namesigdate}[2][5cm]{%
  \begin{tabular}{@{}p{#1}@{}}
    #2 \\[2\normalbaselineskip] \hrule \\[0pt]
    {\small \textit{Signature}} \\[2\normalbaselineskip] \hrule \\[0pt]
    {\small \textit{Date}}
  \end{tabular}
}

\usepackage[T1]{fontenc}
\usepackage[letterpaper, portrait, margin=0.75in]{geometry}
\usepackage[singlespacing]{setspace}
\usepackage{url}
\usepackage{tocloft}
\usepackage{listings}
\usepackage{color}
\usepackage{pgfgantt}
\setlength{\parindent}{0pt}

\begin{document}
\title {Requirements for Noctilucent VR}
\author {Taylor Fahlman, Joshua Bowen, Adam Puckette}

\maketitle

\abstract

\newpage
\thispagestyle{empty}
\mbox{}

\section{Overview}
\subsection{Scope}

This document describes the software design and information regarding the Virtual Reality Lidar Point Cloud Viewing Software known as Noctilucent VR.
This software is intended to be used by professionals seeking to improve their Lidar Cloud viewing experience.
Uses can include anything from casual cloud data viewing to professional informatics.

Noctilucent and it's design can be used on any computer capable of handling the graphics rendering.
This projects only applicability restriction is in the capabilities of the computer running it.

\subsection{Purpose}

Noctilucent VR exists to expand the current capabilities of humans to view Lidar Cloud data.
Specifically the goal of this project is to take 3D information previously displayed in 2D and provide an easy to use, easily accessible, Open Source alternative to viewing this data which allows for it's viewing in 3D.
This project exists as viewing 3D data in 3D is an important aspect of being able to garner easily all the information from that data as possible.

\subsection{Intended Audience}

This project is primarily intended to be used by scientists, engineers, and those who perform informatics, however, the project is not limited to those audiences.
The Free and Open Source nature of this project allows it's use by anyone with access to cloud compare, a virtual reality headset, a wiimote, and a computer capable of running this software.

\subsection{Conformance}

Noctilucent VR conforms to this document if it satisfies the requirements as they are outlined in Sections 4 and 5 of this Document. Requirements are denoted by the verb shall.

\section{Definitions}

Free and Open Source Software (FOSS)

Any software that can be classified as both free software and open source software.
This means that anyone is freely licensed to modify, copy, or stufy the code as they see fit.
Users are encouraged to improve the software and share their work with others.

Virtual Reality (VR)

An Existing technology which allows the user to put ona headset and have an immersive experience in which turning your  head turns the camera and mmoving your head moves the camera.
Visuals are rendered seperately for both eyes allowing for different perspectives which come together to create a 3D stereoscopic environment.

CloudCompare (CC)

A FOSS program freely available to users who need to view and analyze point cloud data.

Open Source Virtual Reality (OSVR)

An existing Open Source VR Headset and VR Software platform.
OSVR is designed so that it can be developed on and modified by the public at only the cost of the headwear.

Graphics Processing Unit (GPU)

The piece of hardware inside a computer responsible for performing microprosses often associated with graphics rendering or deep learning.

Frames Per Section (FPS)

The refresh rate at which the computer draws a scene over the course of a second.

\section{Conceptual Model}

\subsection{Introduction}
\subsection{SDD Idenfication}
\subsection{Design Stakeholders}
\subsection{Design Views}
\subsection{Design Viewpoints}
\subsection{Design Elements}
\subsection{Design Overlays}
\subsection{Design Rationale}
\subsection{Design Languages}

\section{Design Description}

\section{Design Viewpoints}

\subsection{Introduction}
\subsection{Context Viewpoints}
\subsection{Composition Viewpoint}
\subsection{Logical Viewpoint}
\subsection{Dependency Viewpoint}
\subsection{Information Viewpoint}
\subsection{Patterns use Viewpoint}
\subsection{Interface Viewpoint}
\subsection{Structure Viewpoint}
\subsection{Interaction Viewpoint}
\subsection{State Dynamics Viewpoint}
\subsection{Algorithm Viewpoint}
\subsection{Resource Viewpoint}

Annexs

\end{document}

