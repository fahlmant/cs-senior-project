\documentclass{article}

\usepackage[T1]{fontenc}
\usepackage[letterpaper, portrait, margin=0.75in]{geometry}
\usepackage[singlespacing]{setspace}
\usepackage{url}
\usepackage{tocloft}
\usepackage{listings}
\usepackage{color}
\usepackage{pgfgantt}
\setlength{\parindent}{0pt}

\begin{document}
\title {Tech Review for Noctilucent VR - Team 53}
\author {Taylor Fahlman}

\maketitle
\newpage
\thispagestyle{empty}
\mbox{}


\section{Introduction}

I (Taylor), am in charge of the selection and implementation of the software component of the project.
Regardless of the software used for viewing data, some framework needs to be used to integrate the Virtual Reality component.
The three main frameworks in use today are the Open Source Virtual Reality framework, the Oculus Rift framework, and the 
HTC Vive framework.
There also needs to be a framework for the UI. This is highly dependent on the software we select to view the data,
but the benefits of each will still need to be explored.

\section{Virtual Reality Development Framework}
\subsection{OSVR}
The Open Source Virtual Reality project has two major components: a headset (hardware) and a framework (software) to enable
other software to use various headsets. Here, the focus in on the viability of the software component of OSVR. 
One big benefit to using the OSVR framework comes from the title; open source. This means that the source code
of OSVR is open and free for anyone to use, change, and contribute to. Specifically for our project, this is helpful
as there could arise some limitation or feature that is needed that other frameworks do not provide. In that case,
writing our own code of top of OSVR, and possibly integrating it back into the core is easy to do. Another benefit
is that it works with a variety of headsets, and it is supposedly easy to add support for new headsets.
The biggest downside is that while there is an official headset for OSVR, it is still not the most robust framework.
There are many improvements that can be made still, and the lower overall quality of this technology shows in demos
of both the software and hardware. Specifically for the software, just getting a demo to work was complicated. The firmware
also needed to be upgraded, which is a technical and long process, and there have been reports of this bricking various
hardware devices. As well, it does not support, for example, the Oculus Rift as well as the Oculus SDK does.

\subsection{Oculus Rift SDK} 

The Oculus Rift SDK is a software development kit intended for the Oculus Rift headset.
The main benefits of this software is that it is professionaly made and supported, with
a very high standard. It also is integreatable with software such as the Unreal game engine,
and the Unity game engine. It also has a large community to learn from and share with. However,
there are several drawbacks. First of all, this is designed for gaming applications mostly. While most
VR is as of today, there exists some limitations in the library for the kind of data analysis required of
this project. The software that the Oculus already works with are not ones this project is considering,
so this does not give the SDK an advantage per say. The largest drawback, however, 
is that this SDK is only targeted and provides official support for the Oculus Rift. 
Other headsets would be difficult or imposible to use with the Oculus Rift SDK.
Due to the constraints and scope of the Oculus Rift SDK, it does not seem like the best choice for the
project. 

\subsection{HTC Vive SteamVR SDK}

The HTC Vive SteamVR SDK is also a software development kit, centered around the HTC Vive headset
and SteamVR by Valve. The SDK is part of the larger HTC OpenSense SDK. The SDK is intended to be used
with the HTC Vive headset,  

\section{UI Framework}
\subsection{Qt}
\subsection{HTML5/CSS/JS}
\subsection{Other}

\section{Sources}

\end{document}

